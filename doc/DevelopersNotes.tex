\documentclass[dvips, 11pt, a4paper]{article}
\usepackage{amsmath}

\addtolength{\evensidemargin}{-1.5cm}
\addtolength{\oddsidemargin}{-1.5cm}
\addtolength{\textwidth}{3cm}
\addtolength{\topmargin}{-2cm}
\addtolength{\textheight}{2cm}
\setlength{\footskip}{3cm}

\begin{document}

\title{Development Notes for the C4 Code} \author{Richard Ferrier}
\date{February 2010}

\maketitle

\section{Fortran2003 and Object Oriented Programming}
\label{sec:Fortran2003AndObjectOrientedProgramming}

Compared with popular OO codes such as C++ and Java, there are a
number of idiosyncracies in Fortran 2003 that one needs to be aware
of.

In the following discussions, references will be made to the different
kinds of class in a class hierarchy.  I will explain them here briefly. 
Going from top to bottom, they
are pure abstract base classes (PABCs, or simply ``interfaces''); impure
abstract classes; and concrete classes.  A PABC has no data or
algorithms; it is simply a set of method signatures that must be
filled in by the classes extending it.  Signatures are called deferred
type-bound procedures in Fortran.  They are defined using abstract
interface blocks.

An impure abstract class unifies several concrete classes by having
their common data and methods written into it.  It may bind some of
the signatures to concrete methods, but it may also defer them to the
concrete classes.  There is a quirk or bug in the Intel Fortran
compiler in this situation, by the way.  The linker can appear to
 fail when an intermediate
abstract class does not provide a binding for a deferred signature.  
A re-declared signature in the intermediate AC appears to fix
this.

A class deemed abstract cannot be used as a basis for
objects since it is incomplete in its implementation.  Concrete
classes, having filled in all remaining signatures, can.  They are not
necessarily the last word in the hierarchy and can be extended just
like abstract classes, optionally having their existing bindings
overridden by new concrete methods.  This is considered 
bad OO practice except in certain cases, such as overriding factory
methods to create fake components when testing.  More on this later.


\subsection{Single Inheritance and Interface Realisation}
\label{sec:SingleInheritanceAndInterfaceRealisation}

Fortran 2003 and C++ use the ``extends'' keyword to achieve both
inheritance (extension of concrete and impure abstract classes) and
interface realisation (extension of pure abstract base classes).  In
Fortran, only one supertype can be extended by each class.  C++ on the
other hand allows multiple inheritance and interface realisation.
Multiple inheritance is controversial and we do not consider its
absence in Fortran a particular hindrance.  But realisation of
multiple interfaces is safe and useful, because it allows data types
to be polymorphic in more than one respect.  The inability to realise
more than one interface has not proved to be a fatal limitation yet.

\subsection{No Templates}
\label{sec:NoTemplatesGenerics}

We get round the lack of templates using a preprocessor.

% tk

\subsection{Private Module Entities, Not Private Class Entities}
\label{sec:PrivateModuleEntitiesNotPrivateClassEntities}




\section{Code Structure}
\label{sec:CodeStructure}

\subsection{Interface vs. Implementation}
\label{sec:InterfaceVsImplementation}

It is common practice to separate interface and
implementation.  The implementation classes, containing data and
algorithms, can then be decoupled from one another and simply
reference the interface.  This leads to code that is more flexible and
modular.

In the code we have subscribed partially but not completely to this
practice.  To have everything implement an interface would require a
lot of run-time overhead owing to polymorphism everywhere - that is,
unless the compiler is sophisticated enough to resolve unnecessary
dynamic bindings into static ones.  Given the imperfections in the
Intel Fortran compiler at the time of writing, I would not count on it.

We are interested in polymorphism only when we have data types that
truly need to be interchangeable at run time, or that need to
back-point to types defined earlier in the use-chain without causing a
circular reference.  Otherwise, I prefer to err on the side of procedural
programming in order to keep the code computationally efficient.

So rather than an ``interface'' layer and an ``implementation'' layer
of modules, there is a ``static'' layer and an ``dynamic'' layer.  The
static layer contains a mixture of non-polymorphic types and impure
abstract classes.  All clients, from programs to concrete classes, will
look to this layer when they need to use or collaborate with a class.

The dynamic layer contains concrete extensions to the abstract classes.


\section{Programming}
\label{sec:Programming}


\subsection{Object Creation}
\label{sec:ObjectCreation}

In programs and subprograms, when an object needs to be created, we
have the choice between statically declaring the concrete class using
the ``type'' keyword and allowing for dynamic allocation using
``class''.  The latter is almost always preferred since it allows for
polymorphism.

Even when the top-level client knows exactly which concrete class it
wants to use, it is often the case that the object goes on to form
part of another higher-level object via Dependency Injection.  DI is
explained below.  And in tests, which take place from the point of
view of an intermediate client, we want to test those methods that are
generally visible to other classes, not auxiliary concrete-specific
methods.  So it makes sense to declare the abstract class.

We will define object creation as dynamic allocation followed by
initialisation.  Dependency injection is concerned with the allocation
step.  For the initialisation, since we are dealing with an abstraction,
we will need to use Fortran's ``select type'' construct.

\subsubsection{Initialisation}

Initialisation is one of the few scenarios when I will use this construct.  This construct should be used with caution
because it penetrates through the abstract interface of an object,
exposing its concrete organs.  Essentially it creates a binding
between the caller and the concrete class.  This is bad programming,
 unless the caller exists in the same module.

In initialisation the construct is necessary because the abstract class has
been declared, but we need to temporarily access the concrete components.
Could we get around it by declaring an abstract initialisation signature 
that passes in a common set of arguments?  Not generally.  As concrete
 classes vary, so will their components.  The combination of
 initialising arguments required will often differ from class to class.

Note that the initialisation call is always received by the concrete class,
at the bottom of the class hierarchy.  The method name ``init'' is
 reserved for this purpose.  But if the concrete class has a superclass
 whose data also needs initialising, it must pass on arguments in a
 nested call.  How then to name the superclass' initialisation method?  
To overload init(...) confuses the compiler.  I have chosen to call it
 simply initXYZ, where XYZ is the name of the superclass.  In order to be 
systematic, the chain of calls will never skip a data-less class; that 
class will simply pass on the call.

\subsubsection{Factory classes}

These are useful but not essential.  Once a concrete class has been designed
and coded up, it may be employed any number of times by client classes and
test programs.  The code needed to do the allocation and initialisation 
is not lengthy, but it becomes repetetive.  I will often factor it into a 
``create'' method, together with a corresponding ``destroy'' method, 
bound to a factory class.  Usually the factory class will be in the same module as the concrete class.

Creating a whole new class to create a corresponding concrete class almost seems like overkill.  One might argue that it would be more elegant to add
the create and destroy procedures to Fortran 90-style interface blocks.
But sometimes different concrete classes will have the same creation 
signature, and the compiler cannot be relied upon to make the distinction.
So we make the client choose a factory object in order to
select a particular concrete class.

Factory classes should not be confused with the GoF Factory Method pattern. 
This pattern essentially makes the creation method abstract in order to
 support polymorphic object creation.  It is useful when some class A
is responsible for the creation of another class B, but the concrete type
 of B depends on the concrete type of A, so that we have A1 creating B1,
A2 creating B2, etc.  As far as I can tell it is only applicable when the 
set of initialisation arguments does not change.  I tend not
to use the FM pattern, as we can get better decoupling with Dependency 
Injection anyway.

[EDIT: Actually we could have ``create'' procedures without the factory 
objects, if we append the target concrete class name to ``create''.]

\subsubsection{Dependency Injection}

This is a mechanism that promotes modularity and
reduces coupling.  If some class A contains a polymorphic component B,
it should not bind itself to a particular implementation of B, but
instead contain a placeholder to B's interface.  The client is then
free to compose A using any combination of components it wants.  To do
this, the client dynamically allocates and initialises B before
passing it to A.  Class A can use Fortran's ``move_alloc'' intrinsic
procedure to transfer the allocation to its placeholder.

An alternative creational design pattern is the GoF Abstract Factory
pattern.  This pattern is useful when a class contains several
components related by a theme.  In our code the classes usually
contain one or two polymorphic components at a time and the AF pattern
would be overkill.  Dependency Injection is a simpler and more
intuitive way of achieving low coupling between modules.


\subsubsection{Initialisation of an Abstract Superclass}
\label{sec:InterfaceVsImplementation}


\section{Testing}
\label{sec:Testing}

\subsection{(Dirty Unit Tests}
\label{sec:DirtyUnitTests}




\begin{thebibliography}{}
\bibitem{Murthy09} Murthy JY, Minkowycz WJ, Sparrow EM, Mathur
  SR. Survey of Numerical Methods, in: \textsl{Handbook of Numerical
    Heat Transfer}, 2nd Ed.  John Wiley \& Sons, Inc., Hoboken, 2009.
\bibitem{Luo10a} Luo XL, Lei KB, Gu ZL, Wang S, Kase K.  A novel
  {C}artesian cut cell solver for incompressible viscous flow in
  irregular domains.  \textsl{Int. J. Numer. Meth. Fluids} [In press].
\bibitem{Ferziger96} Ferziger JH, Peri{\'c} M. \textsl{Computational
    Methods for Fluid Dynamics}.  Springer, Berlin, 1996.
\end{thebibliography}

\end{document}
